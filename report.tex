\documentclass[12pt]{report}
\linespread{2}
\usepackage[pdftex]{graphicx}   
\usepackage{amssymb}
\usepackage[utf8]{inputenc}
\usepackage[english]{babel}
\usepackage{natbib}
\usepackage{amsmath}
\usepackage{graphicx}
\usepackage{stackengine}
\usepackage[margin=1.25in]{geometry}
% latex
\usepackage[hidelinks]{hyperref}
%\usepackage{hyperref}
\usepackage[hyphenbreaks]{breakurl}




%opening
\title{Unsupervised Out of Vocabulary Word Handling for Neural Machine Translation.}
\author{Raja Gunasekaran}

\usepackage{graphicx}
\graphicspath{ {images/} }

\begin{document}

\maketitle

%\begin{abstract}
%
%\end{abstract}

\tableofcontents
%\listoffigures
%\listoftables

\chapter{Introduction}
\section{Introduciton}

%NMT systems are well suited for any problem that can be formulated as mapping an input sequence to an output sequence \citep{sutskever2014sequence}.

Machine Translation (MT) is the process of translating text automatically from one natural language to another  language \citep{russell2002artificial}. Until recently Statistical Machine Translation (SMT) approaches like phrase-based translation models \citep{koehn2003statistical} dominated the field of machine translation. They were widely adopted and used in translation engines like Google Translate. Neuro Machine Translation (NMT) is a recent approach to MT using Neural Networks.  \cite{kalchbrenner2013recurrent} proposed the first end to end system using encoder-decoder structure for MT. This led to development of more complex encoder-decoder models like \cite{graves2013generating}, \cite{kalchbrenner2013recurrent}, \cite{sutskever2014sequence}, \cite{cho2014learning}, etc., with additional functionality and improved performance.

These NMT systems use vector representation to represent input words called word embeddings. It is a way of representing words in the vocabulary using vectors in a high dimensional space. These embeddings can be pre-trained on a large monolingual corpus and saved for later use. Embedding models discussed in  \cite{bengio2003neural}, \cite{mikolov2013distributed}, \cite{pennington2014glove}, etc., learn word representations in continuous vector space where similar words occur closer to each other. 


One of the challenges when using pre-trained word embeddings for any natural language processing (NLP) task is the issue of handling out-of-vocabulary (OOV) words. This problem occurs when a word that was unseen during training time occurs in testing phase during translation. This problem is more pronounced while working with morphologically rich languages or low resource languages. In this project, I propose and implement an NMT systems based on \cite{bahdanau2014neural} that is able to handle OOV words online during translation. OOV words will be analysed for their morphology and mapped to an in-vocabulary word using the technique from \cite{soricut2015unsupervised}.

Next section gives the background for the project. In Section 3, related NMT systems that can handle unknown words and their techniques are presented. In section 4, the proposed work and its components are presented. Dataset used in the project and Evaluation metric are presented in section 5. In the last section, the implementation status and timeline for the projects is given.
%
%a classic NMT model from \cite{bahdanau2014neural} and extending the work of \cite{soricut2015unsupervised} for handling OOV words online during translation. 




\chapter{Literature Survey}
%\section{Background}
This section provides a background on two main topics of the project: Recurrent Neural Networks (RNN) and Machine Translation (MT). We begin by looking into neural networks and why RNNs are well suited for MT. Then, we will look into some of the historical and current approaches for machine translation.

\section{Neural Networks}
\cite{rojas2013neural} define neural networks as ``distributed, adaptive, generally nonlinear learning  machines built from many different processing elements''. They can approximate a function by learning the input-output mapping. Neural networks like feed-forward multi-layer perceptrons, shown in Figure \ref{mlp}, can approximate any (continuous) function to an arbitrary accuracy if the number of hidden neurons are large enough \citep{hornik1989multilayer}. These networks can be trained using the backpropogation algorithm \citep{rumelhart1988learning} by updating the weights and biases based on an objective function. As the training progresses, the network updates its weights in a way that its predicted output moves closer to the ground truth\footnote{ground truth - reference output provided by direct observation as opposed to inference}. A one layer feed-forward neural network can be written as follows:


%of the network to match the predicted output to the labelled output
\begin{equation}
NN_{MLP1} (x) = g(xW^1 + b^1 )W^2 + b^2
\end{equation}

where g is any non linear activation function, $x$ is an input vector, $W^1, W^2, b^1$ and  $b^2$ are the weights and biases for the network.

Unlike other machine learning approaches where input features have to be hand-engineered, neural networks can learn the required features from the training data. 

\begin{figure}[ht]
	\centering
	\includegraphics{images/multilayerperceptron_network}
	\caption{Multilayer perceptron network with one hidden layer. Figure taken from  \cite{pedregosa2011scikit}}
	\label{mlp}.
\end{figure}

One class of neural network model called Recurrent Nerual Network (RNN) \citep{elman1990finding} is particularly well suited for machine translation. In natural languages, the input is a sequence of words of arbitrary length based on some structure properties of the language. RNNs allow for representing an input sequence of arbitrary length as fixed-sized vectors based on its structural properties. A simple RNN \citep{goldberg2016primer} can be defined as follows.

\begin{align}
h_{1:n},  y_{1:n} &= RNN(h_0,x_{1:n}) \\
h_i &= R(h_{i-1},x_i;\theta) \\
y_i &= O(h_i;\theta)
\end{align}
\[ 
x_i \in \mathbb{R}^{d_{in}}, y_i \in \mathbb{R}^{d_{out}}, h_i \in \mathbb{R}^{f({d_{out}})} \]

where $x_{1:n}$ is the input vector, $y_{1:n}$ is the output vector and $h_{i}$ is the state vector at time-step $i$. $R$ is a non linear function applied over current input $x_i$ and previous hidden state $s_{i-1}$. $O$ is an additional function applied over the current hidden state to generate output vector.  Parameters $\theta$ are shared across the network. A simple RNN uses $sigmoid$ or $tanh$ as the non linear function in the neural units. Special kinds of neural units like Long Short-Term Memory (LSTM) \citep{hochreiter1997long} or Gated Recurrent Units (GRU) \citep{cho2014learning} can also be used. A graphical representation of the same network is shown in Figure \ref{rnn}


\begin{figure}[ht]
	\centering
	\includegraphics[scale=0.3]{images/rnn}
	\caption{Graphical representation of RNN. Figure taken from \cite{goldberg2016primer}}.
	\label{rnn}
\end{figure}


The input for these neural networks is a sequence of vectors not words. NMT systems use word embeddings to represent input words. It is a way of representing words in a vocabulary as vectors in a higher dimensional space. These representations have been good at capturing semantic and syntactic regularities in languages \citep{mikolov2013distributed}. When used as input, these models have also been shown to improve the performance of many NLP systems in tasks such as \textit{MT} \citep{vaswani2017attention, sennrich2015neural}, \textit{sentiment classification} \citep{kumar2016ask}, and \textit{part of speech tagging} \citep{kumar2016ask}. The vector representation for words in these models primarily depend their co-occurrence count with other words within a window size. These vectors capture syntactic, semantic and morphological properties of the words. 

%\begin{figure}[ht]
%	\centering
%	\includegraphics[scale=0.5]{images/embeddings}
%	\caption{Two-dimensional Principal component analysis (PCA) projection of 1000-dimensional vectors of countries and their capitals \citep{mikolov2013distributed}}.
%	\label{embeddings}
%\end{figure}


%\footnote{Principal component analysis (PCA is dimensionality technique used here to show high dimensional vectors in 2-dimensional space.)}

One of the major challenges for many embedding models is their inability to handle Out Of Vocabulary (OOV) words.  This problem is more pronounced in languages with rich morphology. A word in morphologically rich languages encodes more information (such as gender, number, tense) as compared to morphologically poor languages which rely on word order and context. Recently, many models have been proposed to solve this problem. \cite{sun2016inside} proposed a method to integrate both external contexts and internal morphemes\footnote{Morphemes are the smallest meaningful units in a language.} to learn better word embeddings especially for rare and unknown words. \cite{bojanowski2016enriching} incorporate subword information like character n-grams\footnote{A character n-gram is a sequence of n characters from a word} for learning word embeddings during training. Overall, character embedding models, where vector representations for every character or character n-grams are learned, have be shown to generate good word embeddings for rare and unknown words. 

%%Most of these models were originally developed in English - a language with strict word order and relatively poor morphology. These models perform the best in English. 
%

%
%
%Despite their advantages, NMT systems are not capable of translating rare and unknown words. In this project, I am implementing an end to end encoder-decoder system for machine translation and an unsupervised method to handle unknown and rare words.
%

\section{Statistical Machine Translation}
Machine translation is a task of translating a source language sentence $F$ to the target language sentence $E$ using computing resources. It can effectively remove language barriers between humans, allowing assimilation of content from different languages. This potential of MT has led to substantial amount of research since the advent of digital computing. There have been many approaches to the problem such as rule-based MT, phrase-based MT, NMT, etc. In the early days, rule based approaches that used dictionaries, grammar and pre-defined rules to translate text were explored until the ALPAC report in 1966 \citep{national1966language}. The ALPAC report showed that post-editing machine translation was not cheaper or faster than human translation. 

%\subsubsection{Statistical Machine Translation}

In the late 1980s, following the success of statistical method on speech recognition, IBM research \citep{brown1993mathematics} modelled the problem of translation as a statistical optimization problem. Many SMT approaches such as word-based models \citep{brown1993mathematics}, phrase-based models \citep{koehn2003statistical,marcu2002phrase}, hierarchical phrase-based  models \citep{chiang2007hierarchical} and syntax based models \cite{galley2004s,galley2006scalable} were proposed. The goal of all the SMT systems is to maximize the probability of target sentence $f$ given the source language sentence $e$

\begin{equation}
\underset{f}{argmax \  } P(f|e) = \underset{f}{arg max } (P(f) \times P(e|f)) 
\end{equation}


where $P(f)$ is a language model and $P(e|f)$ is a translation model. Language model and translation model are sub-components of SMT systems. Language models learn to assign probability to a sequence of words in a language. They assign higher probability to sentences that are more likely to occur in the language and hence a measure of fluency in the language. Language modelling is central to many tasks in NLP including MT.  

Translation models learn the mapping between source and target language words or phrases. They measure word level translation accuracy between source and target sentences. These models were built by analyzing  monolingual, bilingual corpus and learning their probability distribution. The rise of digital text resources like parallel corpora and an increase in computing power and storage, fuelled the growth of SMT systems. Although SMT systems are robust to noisy data,  they required fine tuning for many components such as language model, reordering model, and translation model for each language pairs. They also require large amount of data and do not handle long range dependencies well.


\section{Neural Machine Translation}

An NMT system is a neural network that models the conditional probability $p(y|x)$ of generating a target language $y$ sentence given the source language sentence $x$. Generally, any NMT system consists of two components i) an \textit{encoder} that computes a sentence representation vector from the source sentence, and ii) a \textit{decoder} that transforms the vector representation to a target language sentence. RNNs are a common choice of network for both encoders and decoders as they process input in a sequential fashion. Convolution neural networks have also been used especially as an encoder \citep{kalchbrenner2013recurrent}.

%In 2003, \cite{bengio2003neural} proposed probabilistic language model based on neural networks which laid the foundation for the use of neural networks in machine translation. 

Initially, neural networks were used as a component in phrase based systems to score the quality of translation \citep{schwenk2012continuous} and to provide additional features to SMT systems \citep{zou2013bilingual}. \cite{kalchbrenner2013recurrent} proposed the first end to end approach for NMT using convolution neural networks as the encoder and RNN as the decoder. Their network suffered from the problem of vanishing gradients where the network was unable to capture long range dependencies. To overcome this problem, more sophisticated activation functions such as LSTM \citep{hochreiter1997long} or GRU \citep{cho2014learning} are used.  \cite{sutskever2014sequence} and \cite{cho2014learning} demonstrated that these gated units can handle long range dependencies better than a simple RNN \citep{elman1990finding}. An encoder-decoder network with LSTM RNNs is shown in Figure \ref{seq2seq}. The boxes in the figure are LSTM RNN units. Mathematically, an LSTM RNN unit is defined in \cite{goldberg2016primer} as follows:

\begin{align}
s_j = R_{LSTM}(s_{j-1}, x_j) &= [c_j:h_j] \\
c_j &= c_{j-1} \odot f + g \odot i \\
h_j &= tanh(c_j) \odot o \\
i &= \sigma(x_j W^{xi} + h_{j-1} W^{hi})\\
f &= \sigma(x_j W^{xf} + h_{j-1} W^{hf})\\
o &= \sigma(x_j W^{xo} + h_{j-1} W^{ho})\\
g &= tanh(x_j W^{xg} + h_{j-1}W^{hg})\\
y_j &= O_{LSTM}(s_j)
\end{align}
\[ 
s_i \in \mathbb{R}^{2 \cdot d_{h}}, x_i \in \mathbb{R}^{d_{x}}, [c_j, h_j, i, f, o, g] \in \mathbb{R}^{{d_{h}}}, W^{xo} \in \mathbb{R}^{d_x x d_h}, W^{ho} \in \mathbb{R}^{d_h x d_h}  \]

where $x_j$, $y_j$ and $h_j$ are the input vector, output vector and the hidden vector repectively. $\odot$ is component-wise product. $i$, $f$ and $o$ are input, forget and output gates respectively. $g$ is the update candidate.

\begin{figure}[ht]
	\centering
	\includegraphics[scale=0.9]{images/seq2seq}
	\caption{RNN with LSTM encoder and decoder. Figure taken from \cite{farizrahman4u}}.
	\label{seq2seq}
\end{figure}


\subsubsection{Attention based Encoder-Decoder models}

These simple encoder-decoder networks summarize the source sentence in a fixed length context vector. \cite{bahdanau2014neural} noted that these networks were inadequate to represent long sentences due to the fixed dimension of the context vector. While this problem could theoretically be solved by increasing the dimension of the context vector, the computing power and memory required to train such a network sets an upper limit even today.

\cite{bahdanau2014neural} proposed an attention based encoder-decoder architecture to mitigate this problem.  This allowed for the encoder to produce better sentence representation for longer sentences which in turn improved the translation quality. In their additive approach, a single feed forward neural networks that can learn to assign different weights to the hidden layer vectors was used. These weighted sum of the hidden layer vectors $h_{1:n}$ called context vector $c_i$ is calculated each time decoder generates a new word as shown in Figure \ref{attention}.

\begin{align}
c_i &= \sum_{j=1}^{n} \alpha_{ij} h_j \\
\alpha_{ij} &= \frac{\hat{a}_{ij}}{\sum_j \hat{a}_{ij}}\\
\hat{a}_{ij} &= att(s_i, h_j)
\end{align}


%\cite{bahdanau2014neural} proposed an attention based encoder-decoder architecture which is capable of learning word alignment between source and target sentences.In their additive approach, a single feed forward neural networks that can learn to assign different weights to the hidden layer vectors was used. 


where $att(s_i, h_j)$ is an attention function that calculates the weights for each encoder hidden state $h_{1:n}$, for a given decoder state $s_i$. They also used bi-directional RNN which reads the sentence from both directions. The state vector from both direction right to left  $\overleftarrow{h_j}$ and left to right $\overrightarrow{h_j}$ is concatenated for each word. The attention mechanism is applied over this concatenated hidden state vector $h_j = [\overleftarrow{h_j};\overrightarrow{h_j}]$. The whole network is trained with negative log-likelihood as the objective function using stochastic gradient descent.


\begin{figure}[ht]
	\centering
	\includegraphics[scale=0.35]{images/attention}
	\caption{Attention based encoder. Figure taken from \cite{bahdanau2014neural}}.
	\label{attention}
\end{figure}


%\begin{align}
%a_t(s) = align(h_t, \bar h_s)  = \dfrac{exp(score(h_t, \bar h_s))}{\sum_{s'} exp(score(h_t, \bar h_{s'}))}
%\end{align}

\cite{luong2015effective} proposed two simpler but effective variations of attention mechanism namely: 1) a global attention model where all words in the source sentence are attended, and 2) a local attention model where only a subset of the source words are attended at a time. In the global attention mechanism, which is very similar to one proposed in \cite{bahdanau2014neural}, they did away with bi-directional, concatenated state vector $h_j = [\overleftarrow{h_j};\overrightarrow{h_j}]$. They simplified the computation path to make it run faster. Their model produced state-of-the-art results in WMT'14 and WMT'15 for English to German translation. 


In this project, we use the global attention based encoder-decoder model proposed in \cite{luong2015effective} as the baseline NMT system for evaluation. The models discussed so far cannot handle words that are not in their vocabulary. In the next chapter, we will look at some of techniques that allow us to handle OOV words.




% Basics needed to understand the rest of the text with references to authoritative literature sources

%
%
%
%
%In their model, the source language sentence will be encoded into a real valued vector using the encoder. The decoder will decode the vector to a target language sentence to produce the output. 
%
%Unlike, SMT systems, NMT systems generalize well, do not require domain knowledge and are more tolerant to noisy data \cite{bibid}. 
%
%
%NMT systems use vector representation, also called word embeddings, for input words and internal states. Word embedding is a way of representing words in the vocabulary using vectors in a high dimensional space.Word embedding models like \cite{bengio2003neural,mikolov2013distributed,pennington2014glove} etc., learn word representation in continuous vector space where similar words are expected to occur closer to each other. These models have been successfully used in many NLP tasks like Machine Translation. The vector representation for word in most of these models primarily depend on word co-occurrence or context words within a window size capturing syntactic, semantic and morphological properties of natural languages. 
%
%
%%Most of these models were originally developed in English - a language with strict word order and relatively poor morphology. These models perform the best in English. 
%
%One of the major challenges for many embedding models is their inability to handle out of vocabulary words.  This problem is more pronounced in languages with rich morphology. A word in morphologically rich languages encodes more information (such as gender, number, tense) as compared to morphologically poor languages which rely on word order and context. 
%
%
%Despite their advantages, NMT systems are not capable of translating rare and unknown words. In this project, I am implementing an end to end encoder-decoder system for machine translation and an unsupervised method to handle unknown and rare words.
%






%In 2003, \cite{bengio2003neural} proposed a language model developed using neural network.
%
%Data Sparcity Problem.


\chapter{Related Work}

NMT systems typically have a vocabulary size of 30,000 to 80,000 words. Until recently, these models were incapable of translating rare and unknown words \citep{luong2015addressing}. As the vocabulary size of the network grows, the complexity and the number of parameters to tune increases rapidly. Pioneering works in NMT  \citep{sutskever2014sequence,bahdanau2014neural} observe that sentences with rare and unknown words often produced poor translations when compared to sentences with many frequently used words. Since the network has seen these words only a few times, they don't get a vector representation that can capture their semantics and morphology. 

\cite{luong2013better} note that rare and unknown words are usually morphologically rich in nature in comparison to more frequently occurring words. Although the problem of handling unknown and rare words was focused heavily in the context of generating word representations, it was not addressed in any of the early works in NMT \citep{luong2015addressing}. 
%In 2015, \citeauthor{sennrich2015neural} proposed a approach to make open-vocabulary translation possible in NMT. They split rare and unknown words as subword units based using Byte Pair Encoding(BPE). 
In this chapter, we present some of the general techniques for handling rare and unknown words. First, we will discuss how this problem is handled in the context of word embeddings. Then, we will explore NMT specific approaches to handle OOV words.


\section{Vector representation for OOV words}

Representing words as vectors has a long history in NLP. A good vector representation should capture the syntax and semantics of a word. It should mirror the linguistic relationship between the words in the vector space. Embedding models discussed in  \cite{bengio2003neural}, \cite{collobert2008unified}, \cite{mikolov2013distributed}, \cite{pennington2014glove}, etc., learn word representations in a continuous vector space where semantically similar words occur closer to each other. Use of word representation, learned from these models, have been shown to improve performance across all NLP tasks \citep{kumar2016ask}. These systems are usually trained on a large monolingual corpus with billions of sentences. Then, the learned word embeddings can be used to represent individual words in the downstream tasks like sentiment classification, text summarization, machine translation, etc.


These models work at word level and ignore the internal structure of the words. Only the words that the model has seen in the training data will get a vector representation. To get vector representations for OOV words,  morphological and orthographic information can be used \citep{botha2014compositional,luong2013better,bhatia2016morphological} while training. These models require an external morphological analyzer to segment the word which is not readily available for all languages. \cite{soricut2015unsupervised} proposed an unsupervised, language agnostic method to extract morphological rules and to build a morphological analyzer. In their approach morphological transformations can be captured and used to map OOV words to an in-vocabulary word.


An alternate approach is to get embeddings for characters or character n-grams by breaking down words \citep{bojanowski2016enriching,kim2016character,wieting2016charagram}.  This allows us to get word embeddings for any words by convoluting over character embeddings. Among these models, \citeauthor{bojanowski2016enriching} showed that, by incorporating character n-grams their models can outperform other word level or morphology based approaches. They learn representations for character n-grams and represent word as the sum of character n-grams. 


For this project, we have used the work of \cite{soricut2015unsupervised} and \cite{bojanowski2016enriching} to represent OOV words. we have implemented and studied the morphological analyzer from \cite{soricut2015unsupervised} on a word similarity task and used the pretrained word representations from \cite{bojanowski2016enriching} for translation task. 


%\begin{table*}[]
%	\centering
%	
%	\begin{tabular}{|l|c|c|}
%		\hline
%		\multicolumn{1}{|c|}{\textbf{Embeddings}}                                          & \textbf{Method}               & \textbf{\begin{tabular}[c]{@{}c@{}}Spearman Correlation $\rho \\ on Stanford  RW dataset\end{tabular}} \\ \hline
%		\cite{soricut2015unsupervised}                                                               & Skip Gram + Morphology        & 0.41                                                                                               \\ \hline
%		\begin{tabular}[c]{@{}l@{}}\cite{bojanowski2016enriching}\end{tabular} & Subword Information Skip Gram & 0.48                                                                                               \\ \hline
%		\cite{sun2016inside}                                                                   & Context + Morphology          & 0.52                                                                                               \\ \hline
%		\cite{barkan2017bayesian}                                                                      & Bayesian Skip-Gram            & 0.50                                                                                               \\ \hline
%		\cite{sanu2017word}                                                                 & FOFE+SVD                      & 0.50                                                                                               \\ \hline
%	\end{tabular}%
%	\caption{State of art performance of different models on RW dataset}
%	\label{my-label}
%\end{table*}

%To get a good representation for each word the system has to be trained on large dataset of million of sentence.

%Across all NLP tasks it been shown that representing words in a continuous vector space as 
%These systems are only as good as the data they are trained on. While English represents only 9\% of the world population, 54\% of the digital content is in English.
%This   These word vector can also be trained on a large monolingual corpus and reused other tasks like machine learning.

%Another disadvantage of improving the vocabulary size, is that  


%The focus of this project will on the performance of NMT systems under low resource settings for morphologically rich languages.
%
%One of the challenges when using pre-trained word embeddings for any natural language processing (NLP) task is the issue of handling out-of-vocabulary (OOV) words. This problem occurs when a word that was unseen during training time occurs in testing phase during translation.  In this project, I propose and implement an NMT systems based on \cite{bahdanau2014neural} that is able to handle OOV words online during translation. OOV words will be analysed for their morphology and mapped to an in-vocabulary word using the technique from \cite{soricut2015unsupervised}.




\section{Handling OOV words in NMT}
%Since 2013, there have been many techniques proposed for NMT to get good vector representations for rare and unknown words.
Unlike Statistical approaches, NMT systems have a fixed vocabulary due to computational complexity of the model. This forces us to handle words outside this fixed vocabulary using some other techniques. Initially, almost all the NMT systems represent all the OOV words using an unknown token \textit{unk}. Later, different approaches were introduced to address this problem. They can be broadly classified into two groups: Dictionary back-off models, and Subword models.

\subsection{Dictionary back-off Models}

\cite{jean2014using} proposed a method based on importance sampling to use a very large target vocabulary without increasing the complexity of the NMT system. In their attention based NMT, the attention weights were used to determine the alignment of unknown (\textit{unk}) target words with the corresponding source word; usually a rare word, in the translation. Then a dictionary is used to replace the \textit{unk} tokens in the target languages with the translation of the rare source word.

\cite{luong2015addressing} proposed a similar model using an external aligner instead of using the attention mechanism. An external aligner was used to align words from the source sentence and the generated target sentence. During training, for all the unknown source words, the aligned target word is replaced with an unknown token \textit{unk}. In their copy attention model, each unknown target word is assigned individual \textit{unk} token based on their source word. The alignments between source words and target words are maintained. In their positional model, a pre-build dictionary is used to replace the unknown source words in the target side, based on the alignment. Both these approaches were effective and showed a better performance than the state-of-the-art in English-French and English-German language pairs. \cite{choi2017context} extended the work of \cite{luong2015addressing} to include multiple positional unknown tokens for digits, proper nouns and acronyms instead of just one \textit{unk} token. 

\subsection{Subword and character Models}
One problem with these dictionary back-off methods is that there is not always 1-1 correspondence between words from different languages because of the variance in degree of morphological synthesis between languages. 

\cite{luong2016achieving} presented a open vocabulary NMT system based on word and character embedding models using RNN. In their hybrid approach, the network translates at word level and falls back to character components for rare words. The representation for rare words is computed using Recurrent Neural Network working on the character level. Their system is faster and easier to train unlike other NMT systems using purely character representations. They produced state-of-the-art for English-Czech translation on WMT'15 dataset.

\cite{sennrich2015neural} proposed a system that works on subword level instead of word level like the previous models. The words are segmented into subword units using Byte Pair Encoding (BPE) \citep{gage1994new}. BPE is a data compression technique, where the most frequency character or character sequences are iteratively replaced with unused bytes. The vocabulary of their NMT system comprises entirely of these subword units of different lengths. 

They demonstrated that subword models achieve better accuracy in translating rare words. The model was able to generate new unseen words during testing time and improved English-German and English-Russian translation over other back-off dictionary models. One of the major contribution of \citeauthor{sennrich2015neural}'s paper was showing that the NMT systems are capable of achieving open vocabulary translation by modelling sub word units. This technique has been used widely to improve translation quality. 

BPE does not split words based on their morphology. So the morphological information that the network already learned is not used. In the next chapter, we present how this information can be used to improve the quality of machine translation.




\chapter{Related Work}
% Attention based Encoder-Decoder architectures for NMT proposed in \cite{bahdanau2014neural} and \cite{luong2015effective} are discussed. In the next section, I discuss about unsupervised morphological induction technique from \cite{soricut2015unsupervised}.

%In this chapter, we describe in detail about the problem of translation morphologically rich languages and the proposed vocabulary expansion technique as a solution to the problem. First, we present why it is difficult to translate morphologically rich language. Then, the proposed approach and its components are presented. Finally, the architecture and implementation details of the different modules is presented. 

%This chapter discusses about the objective of this project and the modules that were implemented. First the objective of the project is presented. Then, we will look into various components of the project; their architecture; implementation details. In the last section, I talk about the vocabulary expansion method that I am studying in this project.

%the vector representation for all words can be generated using external monolingual corpus. Using this fixed word vectors as input, the NMT systems are trained under limited vocabulary.



In this Chapter, we present our proposed approach for improving the quality of machine translation for morphologically rich languages. In Section \ref{sec:problem}, we present the challenges in translating a language with rich morphology. In Section \ref{sec:propose}, we describe in detail the proposed vocabulary expansion technique as a solution to the problem. All the components implemented for this study and their architectures are presented in Section \ref{sec:components}. Finally, in Section \ref{sec:implementation}, the implementation details of all the modules are discussed.


\section{Problem}
\label{sec:problem}
NMT systems have been very successful in achieving state-of-the-art performance for many language pairs such as English-French and English-German. However, these systems are not capable of translating rare and unknown words \citep{luong2015addressing}. All NMT systems have a fixed vocabulary of 30k-80k most frequently occurring words. So, words that do not occur or occur rarely in the training data will not get good vector representations. This results in poor translation performance on sentences with rare and unknown words (a.k.a OOV words) \citep{sutskever2014sequence,bahdanau2014neural}. 


This problem with translating OOV words is more pronounced in morphologically rich languages like Turkish, German, Tamil etc. There are two factors that primarily contribute to this:
\begin{itemize}
	\item In morphologically rich languages, due to inflections, there are many possible \textit{word forms} per lexeme as shown in Table \ref{turkish}. For example, even in a morphologically impoverished language like English, we have \textit{run, running, ran, runs} which are all the forms of the same lexeme \textit{run}. So, the vocabulary size in these languages is very high resulting in \textbf{more OOV words}. 
	
	\item NMT requires a large amount of training data for higher performance. Many of these languages have a much  \textbf{less data} available for training a translation system. For example, there are only 300k sentence pairs available for Turkish-English and approximately 200k sentence pairs available for Tamil-English. 
\end{itemize}

To overcome these problems, we need a system that can handle OOV words effectively by understanding the morphology of the words. This system should be capable of mapping the OOV words to an in-vocabulary word using morphological transformations. So, to address this problem, we need to answer the following questions:

\begin{enumerate}
	\item How do we get vector representations for OOV words, that captures the linguistic properties of a word?
	\item Can we improve the performance of NMT systems using vector representations that exhibit linguistic regularities?
\end{enumerate}

%Two factor co where the available data is sparse compared to vocabulary size in these languages. 

%This required handling rare and unknown words in translation systems, particularly under low resource settings. 

\section{Proposed Vocabulary Expansion Technique}
\label{sec:propose}
The main objective of this project is to improve the quality of machine translation for morphologically rich languages by incorporating morphological information. In particular, we focus on improving the translation quality under low-resource settings - when only a smaller amount of data is available for training. To address this problem in NMT systems, we propose and study a simple source vocabulary expansion technique that can translate any source word. 

Word embedding models proposed in \cite{mikolov2013distributed,pennington2014glove}  have been shown to exhibit linguistic regularities in the form of semantics  and morphology as shown in eq 4.1 and eq 4.3 \citep{soricut2015unsupervised}. 

\begin{align}
\vec{\textit{man}} - \vec{\textit{woman}} + \vec{\textit{king}} &  \approx \vec{\textit{queen}} \\
\vec{\textit{stop}} + (\vec{\textit{walking}} - \vec{\textit{walk}}) & \approx \vec{\textit{stopping}} \\
\vec{\textit{stop}} + \vec{(\textit{suffix:ing})} & \approx \vec{\textit{stopping}} 
\end{align}


Inspired by this observation, we hypothesize that fixed word embeddings can be used to expand the source side vocabulary in any NMT system. Instead of letting the NMT system learn word representation, we use a \textit{fixed} pre-trained word representation during training. During training, the NMT system learns the mapping from source language word embeddings to target language word embeddings. 

By fix the word embedding, we achieve the following benefits:
\begin{itemize}
	\item We retain the linguistic regularities present in word representations and feed them to a translation system.
	\item We reduce the number of parameters to tune for NMT systems resulting in faster training.
\end{itemize}


During testing, for any OOV source language word, we generate a word representation which maintains the linguistic regularities. Consider this example: We have a vector $\vec{(suffix:ed)}$ that can transform all the present tense words to their corresponding past tense word in the vector space. If we encounter an OOV word like \textit{$\vec{enthralled}$}, we can use the vector for the in-vocabulary word \textit{$\vec{enthrall}$} and $\vec{(suffix:ed)}$ to generate a meaningful vector representation. This vector can now be fed into the NMT system instead of mapping it to \textit{unk} word. There have been many approaches proposed that make use of pre-trained word embeddings. But, to the best of our knowledge, vocabulary expansion based on fixed word embedding have not been studied.

%this is the first time a vocabulary expansion technique based on fixed word embeddings is proposed.

%To get good vector representation for OOV words, I have compared different approaches for it on a standard data-set for their performance. The best performing models is then taken and used as an input 


To generate these morphological transformations, we experiment with two models: 
\begin{itemize}
	\item Language agnostic, unsupervised morphological transformation approach from \cite{soricut2015unsupervised}
	\item Word representation using subword information from \cite{bojanowski2016enriching}
\end{itemize}

From our experiments, discussed in next chapter, we find that \citeauthor{bojanowski2016enriching}'s approach generates a more accurate vector OOV word. We use this model in our final NMT system. To study the effectiveness of the proposed approach, we implement this technique on a baseline model and report the performance gains. For baseline NMT systems, we implement the global-attention based encoder-decoder RNN model from \cite{luong2015effective}. 

%They use a soft attention model that determines which words to focus in source language while generating target language words.

%For morphological analyser, I will extend the work of \cite{soricut2015unsupervised} where they proposed an language-agnostic, unsupervised method for inducing morphological transformation between words.


\section{Project Components}
\label{sec:components}
In this section, we discuss the baseline NMT systems and the morphological analyzer implemented for this project. The architecture of these two systems is explained in detail below.
%For this project, we implemented and used the following system:
%\begin{itemize}
%	\item An global attention based NMT system \cite{luong2015effective}.
%	\item Morphological analyzer proposed in \cite{soricut2015unsupervised} to handle rare and unknown words.
%\end{itemize}

%
%In this project, we will focus on the two problems in NMT systems. 
%\begin{enumerate}
%	\item We will study the performance of existing methods
%	in generating vector representation for OOV words based on their morphology. 
%	\item We will demonstrate the effectiveness of the proposed source side vocabulary expansion technique in NMT systems. 
%\end{enumerate}
%

%In this project, I am implementing and studying a simple vocabulary expansion technique for NMT systems, based on word morphology. 


%Although the technique was proposed for lexicon generation, the morphological rules and their vector representations learned by the technique can be used to produce vector representations for OOV words in NMT systems.




\subsection{Global Attention Based Model}
Simple encoder-decoder NMT models \citep{sutskever2014sequence,cho2014learning} do not translate long sentences very well. To overcome this problem, attention-based NMTs are used. Their networks learn word alignments between the source and target language sentence.  \cite{luong2015effective} proposed two simple but effective attention based NMT models as an alternative to the attention model from \cite{bahdanau2014neural}. 

Similar to \citeauthor{bahdanau2014neural}'s approach, their \textit{global attention} model focuses on all the source words for every target word. Their \textit{local attention} model focuses only on a small subset of the source words. For this project, we use the global attention model as the baseline.


\begin{figure}[ht]
	\centering
	\includegraphics[scale=0.5]{images/global_attention}
	\caption{Attention based encoder. Figure taken from \cite{luong2015effective}}.
	\label{global_attention}
\end{figure}

Figure \ref{global_attention} presents a visual representation of the global attention model. The context vector $c_t$ is calculated based on alignment weights $a_t$ for every target word. To calculate attention weights, they experimented with three different scoring functions as shown in eq \ref{luong_score}. They noted that with the global attention model, a simple \textit{dot} product between encoder and decoder hidden layer vectors performs the best. We use this model with \textit{dot product} as the scoring function.

\begin{align}
\label{luong_score}
score(\overrightarrow{h_j}, s_i) =
\begin{cases}
\overrightarrow{h_j} ^\top s_i & dot \\
\overrightarrow{h_j} ^\top \textbf{W}_a s_i & general \\
v_a ^\top \textbf{W}_a [ \overrightarrow{h_j}, s_i ] & concat
\end{cases}
\end{align}


%\cite{bahdanau2014neural} proposed an attention based encoder-decoder architecture which is capable of learning word alignment between source and target sentences. This allowed for the encoder to produce better sentence representation for longer sentence which in turn improved the translation quality. In their additive approach, a single feed forward neural networks that can learn to assign different weights to the hidden layer vectors was used. These weighted sum of the hidden layer vectors $h_{1:n}$ called context vectors $c_i$ is calculated each time decoder generates a new word as shown in figure \ref{attention}.
%
%
%\begin{align*}
%c_i &= \sum_{j=1}^{n} \alpha_{ij} h_j \\
%\alpha_{ij} &= \frac{\hat{a}_{ij}}{\sum_j \hat{a}_{ij}}\\
%\hat{a}_{ij} &= att(s_i, h_j)
%\end{align*}
%
%where $att(s_i, h_j)$ is an attention function that calculates the weights for each encoder hidden state $h_{1:n}$ for a given decoder state $s_i$. %\cite{bahdanau2014neural} also used bi-directional RNN which reads the sentence from both directions. The state vector from both direction right to left  $\overleftarrow{h_i}$ and left to right $\overrightarrow{h_i}$ is concatenated for each word. The attention mechanism is applied over this concatenated hidden state vector $h_i = [\overleftarrow{h_i};\overrightarrow{h_i}]$. The whole network is trained with negative log-likelihood as objective function using stochastic gradient descent.

\subsection{Unsupervised Morphological Analyzer}

\cite{soricut2015unsupervised} proposed a language-agnostic, heuristic method to capture morphological transformations by exploiting regularities present in word embeddings \citep{mikolov2013distributed}. Their method automatically deduces morphological rules and transformations from word vectors. These morphological transformations can be represented as vectors in the same embedding space. During testing, OOV words can be mapped into the same vector space using the learned morphological transformations. In this algorithm, the morphological rules are learned as follows.


\begin{figure}[ht]
	\centering
	\includegraphics[scale=0.5]{images/morph_graph}
	\caption{A part of normalized, directed graph with morphological mapping. Figure taken from  \cite{soricut2015unsupervised}}.
	\label{graph_morph}
\end{figure}

\begin{enumerate}
	\item Extract candidate morphological rules like \textit{('suffix', 'ies', 'y')} (replace suffix \textit{ies} with \textit{y} from the word \textit{treaties} to get \textit{treaty}) from every possible word pairs in vocabulary V. 
	\item Then, evaluate the quality of rules in the pre-trained embedding space by calculating the hit-rate for other word pairs.
	\item Generate morphological transformation from the above candidate rules and build a cyclic multi-graph, where words form the nodes and edges form the morphological transformations. The edges between the words are weighted using cosine distance between their vectors and their ranks $(rank, cosine)$.
	\item From the above graph, build a normalized acyclic graph with 1-1 morphological mapping between words. This graph is normalized by mapping low-frequency words to high-frequency words as shown in Figure \ref{graph_morph}. In this graph, all the low-frequency words in the dataset \textit{(recreates, recreations, creates)} are mapped to higher frequency words \textit{(create, created)}.
	\item Use the final graph and the learned morphological transformations to map the rare/out of vocabulary words in the same vector space.
\end{enumerate}

%\subsubsection{Candidate Rule Extraction}
%For every word pair $(w_1 , w_2 )$ in vocabulary $V$, all possible prefix and suffix substitutions from $w_1$ to $w_2$ of predefined length 6 are extracted. For each rule $r$, its support set $S_r$ - set of all word pair that obey rule $r$, is extracted.
%
%\[ S_r = \{(w_1 , w_2 ) \in V^2 |w_1 \xrightarrow{r} w_2 \} \]
%
%
%\subsubsection{Evaluate candidate rules}
%The hit rate for each rule is computed using cosine rank. For every word-pair combination in $S_r \times S_r$, we can calculate the percent of the time similairity rank is higher than the threshold $t_{rank}$. For word pair $((walking, walk),(talking, talk))$, the rank of $(walking, walk + \uparrow d_{talk\_\_})$ was computed.  It can be seen that meaning preserving rules like $suffix:ed:ing$ receive high hit rate while non meaning preserving rule receive low hit rate.
%
%
%
%\subsubsection{Generate Morphological Transformation}
%
%
%For each rule $r$, the best direction vector $\uparrow{d_w}$ is computed greedily. The direction vector $\uparrow{d_w}$ which explains most of the word pair in support is selected recursively for all unexplained word pairs. By doing this, we will have the morphological transformations for each rule. These morphological transformations are subject to further evaluation using cosine similarity and cosine similarity rank. The cosine similarity threshold was set to $t_{cosine}$ = 0.5 and cosine rank threshold was set to $t_{rank}$ = 30. If these thresholds are not met, the rules are removed.
%
%
%The morphological transformations also have a graph based interpretations. A labelled, weighted,	, directed multi-graph $G^V_{Morph}$ is created from the transformations where nodes are the words in vocabulary and edges are the morphological transformation with $(rank, cosine)$ as the weights. From all the candidate morphological transformations 1-1 mappings are selected based on the the following conditions.
%
%\begin{itemize}
%	\item Edges are considered only if they start from higher count word and end at lower count word.
%	\item If multiple edges exist, the ones with minimal rank $rank$ is selected.
%	\item If multiple edges still exist, the one with maximum $cosine$ is selected.
%\end{itemize}
%
%This normalizes the graph to weighted directed graph $D^V_{Morph}$ where low occurring words are mapped to high occurring words using a direction vector.
%
%
%\subsubsection{Mapping Rare and Unknown Words}
%The word vectors for out of vocabulary words, and low frequency words can generated by mapping them to nodes in $D^V_{Morph}$ using the morphological rules. $V_c$ is the set of all low frequency words and OOV words.
%
%\[ V_c=\{w \in V|C\leq count(w)\} \]
%
%All the morphological rule $r\in R$ are applied on words $w\in V_c$ and if any of them results in a word $w'\in D^V_{Morph}$ then that word is mapped into the graph using the direction vector for rule $r$. This allows us to get meaning full representation for Rare and Unknown words at testing time.


Using this approach, if the word \textit{unassertiveness} occurs in the source sentence and is not found the vocabulary of the word embedding model, we would still be able to get a meaningful vector representation for the word. Traditionally, any word not in the vocabulary is mapped to the token \textit{unk}. Using this approach, we can learn vector representation for morphological transformations like \textit{(prefix,un,$\epsilon$)} and \textit{(suffix,$\epsilon$,ness)}. These morphological rules and their vector representation can then be used to map the OOV word \textit{unassertiveness} to  \textit{assertive} and get a good vector representation. 



\section{Implementation Details}
\label{sec:implementation}
In this section, we will present the implementation details of \textit{attention based NMT} and \textit{morphological analyzer} for this project.

The baseline NMT system was implemented from scratch in \textit{pytorch}, a deep learning framework. Later, we switched to \textit{OpenNMT-py}\footnote{https://github.com/OpenNMT/OpenNMT-py} for the fine-tuned performance gains it offers. OpenNMT-py is a collaborative research-friendly framework focusing specifically on machine translation. 

We implemented the morphological analyzer purely in Python. We used \textit{gensim\footnote{https://github.com/RaRe-Technologies/gensim}} to calculate word similarities and their rank in the vector space. For building and storing graphs, we used \textit{networkx\footnote{https://networkx.github.io}} graph library. Finding the nearest neighbour in 300 dimension vector space, with 3M candidates is expensive. Since we do this calculation many time for each word, it can takes days to run on a standard computer. To speed up computation, we used k-d tree based Approximate Nearest Neighbour (ANN) algorithms from \textit{annoy}\footnote{https://github.com/spotify/annoy} library under low precision settings.

%For finding nearest neighbour in vector space, we used annoy\footnote{https://github.com/spotify/annoy} library. 
%Major chunk of the time was spent on optimizing the run-time and memory usage of the program.

\chapter{Experiments and Evaluation}
\section{Dataset and Evaluation}
The proposed approach will be evaluated on English-German translation using the bilingual, parallel corpora from ACL WMT 2016\footnote{http://www.statmt.org/wmt16/translation-task.html}. The training corpora include Europarl (1.9M sentences), Common Crawl corpus (2.3M sentences) and News Commentary v11 (240,000 sentences). newstest2013 (3000 sentences) and newstest2014 (3000 sentences) data will be used as validation dataset.

To be comparable with other existing NMT systems, I will evaluate the models using standard BLEU score metric from  \cite{papineni2002bleu}. BLEU is one of the most popular and widely used automatic MT evaluation metric. It reports the correlation of machine translations with human translations measured using degree of n-gram overlap. As a baseline model, the attention model from \cite{bahdanau2014neural} will be implemented and used. Then, the proposed OOV word handling approach will be added to the baseline model and the translation performance will be studied. Translation performance will reported on newstest2015(3000 sentences) and newstest2016(3000 sentences) dataset from ACL WMT 2016. If the proposed model shows an improvement in BLEU score, this work can be extended and applied to languages with richer morphology, especially agglutinative languages like Turkish or Tamil.


\section{Experiments}
For this experiments, 
First, do words embeddings trained from machine translation hold similar properties as word embeddings trained for language modelling.
No, they do not hold the same properties.




\chapter{Conclusion and Future Work}


\section{Summary}
Machine translation can aid in overcoming the human language barrier in communication. Current translation systems rely on large amounts of parallel corpora for training. But, most of the languages in the world have only a very small amount data available for training these system. Translating morphologically rich languages is more difficult as compared to morphologically poor languages. 


In this project, our goal was to improve the quality of machine translation for morphologically rich, low-resource languages. We presented a vocabulary expansion technique that improved the machine translation from (German and Turkish) $\rightarrow$ English. We demonstrated that our approach improved the translation performance by approximately 2-3 BLEU points.

In our proposed approach, we use pre-trained, fixed word vectors as our input. By doing so, we separated the problem of handling Out of Vocabulary (OOV) words from NMT systems. Therefore, as long as the word embeddings capture the morphological and semantic information of the words and exhibit regularity in their mapping, the translation systems will be able to use that information to translate OOV words. 

Our approach is independent of the underlying architecture of the base NMT system. This technique can be incorporated in any NMT systems that work at word level instead of character or sub-word level.

\section{Future Directions}

In our approach, we used a simple attention based NMT system from \cite{luong2015effective}, as the underlying base system. The performance in the translation task is limited by the ability of the base system. In future, more experiments, with varying training data sizes can be done on more sophisticated models like purely attention based architecture or convolution neural network (CNN) can be done. Recently, \cite{gehring2017convolutional} proposed a fully parallelizable CNN architecture for translation. \cite{vaswani2017attention} proposed a simple network architecture without using any recurrence or convolution. These two networks can be a very good candidate for a base system.


Another promising direction to explore is the expansion of target vocabulary of NMT systems. If a similar improvement is observed in the translation quality, the training time of the network can be greatly reduced.

%\section{Implementation status and Timeline}

The current status of the project and an estimated timeline for my proposed work has been given in the table \ref{timeline}.

\begin{table}[!htbp]
	\centering
	\resizebox{\textwidth}{!}{%
		\begin{tabular}{|l|c|}
			\hline
			\textbf{Work}                                                                                                                             & \textbf{Expected completion date} \\ \hline
			Proposal defense                                                                                                                          & Feb, 2018                        \\ \hline
			\begin{tabular}[c]{@{}l@{}}Implementation of morphological analyzer \\ \citep{soricut2015unsupervised}               \end{tabular}                                                                                                            & Completed                        \\ \hline
			\begin{tabular}[c]{@{}l@{}}Implementation of NMT system \\ \citep{bahdanau2014neural} \end{tabular}                                                                                       & Completed                      \\ \hline
			Handling OOV in NMT                                                                                                                           & In progress                         \\ \hline
			Result analysis  and report writing                                                                                                                          & Feb - Mar, 2018                         \\ \hline
		\end{tabular}%
	}
	\caption{Thesis Timeline}
	\label{timeline}
\end{table}

%\subsubsection{Candidate Rule Extraction}
%For every word pair $(w_1 , w_2 )$ in vocabulary $V$, all possible prefix and suffix substitutions from $w_1$ to $w_2$ of predefined length 6 are extracted. For each rule $r$, its support set $S_r$ - set of all word pair that obey rule $r$, is extracted.
%
%\[ S_r = \{(w_1 , w_2 ) \in V^2 |w_1 \xrightarrow{r} w_2 \} \]
%
%
%\subsubsection{Evaluate candidate rules}
%The hit rate for the rules is computed using cosine rank. For every word-pair combination in $S_r \times S_r$, we can calculate the percent of the time similairity rank is higher than the threshold $t_{rank}$. For word pair $((walking, walk),(talking, talk))$, the rank of $(walking, walk + \uparrow d_{talk\_\_})$ was computed.  It can be seen that meaning preserving rules like $suffix:ed:ing$ receive high hit rate while non meaning preserving rule receive low hit rate.
%
%
%
%\subsubsection{Generate Morphological Transformation}
%
%
%For each rule $r$, the best direction vector $\uparrow{d_w}$ is computed greedily. The direction vector $\uparrow{d_w}$ which explains most of the word pair in support is selected recursively for all unexplained word pairs. By doing this, we will have the morphological transformations for each rule. These morphological transformations are subject to further evaluation using cosine similarity and cosine similarity rank. The cosine similarity threshold was set to $t_{cosine}$ = 0.5 and cosine rank threshold was set to $t_{rank}$ = 30. If these thresholds are not met, the rules are removed.
%
%
%The morphological transformations also have a graph based interpretations. A labelled, weighted,	, directed multi-graph $G^V_{Morph}$ is created from the transformations where nodes are the words in vocabulary and edges are the morphological transformation with $(rank, cosine)$ as the weights. From all the candidate morphological transformations 1-1 mappings are selected based on the the following conditions.
%
%\begin{itemize}
%	\item Edges are considered only if they start from higher count word and end at lower count word.
%	\item If multiple edges exist, the ones with minimal rank $rank$ is selected.
%	\item If multiple edges still exist, the one with maximum $cosine$ is selected.
%\end{itemize}
%
%This normalizes the graph to weighted directed graph $D^V_{Morph}$ where low occurring words are mapped to high occurring words using a direction vector.
%
%
%\subsubsection{Mapping Rare and Unknown Words}
%The word vectors for out of vocabulary words, and low frequency words can generated by mapping them to nodes in $D^V_{Morph}$ using the morphological rules. $V_c$ is the set of all low frequency words and OOV words.
%
%\[ V_c=\{w \in V|C\leq count(w)\} \]
%
%All the morphological rule $r\in R$ are applied on words $w\in V_c$ and if any of them results in a word $w'\in D^V_{Morph}$ then that word is mapped into the graph using the direction vector for rule $r$. This allows us to get meaning full representation for Rare and Unknown words at testing time.
%


% SUmmarize key points. less emphasis on technical grounds. Put in exact technical content.

\bibliography{bib}
\bibliographystyle{apalike}


\end{document}
